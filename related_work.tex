\section{Related Work}
\label{sec.related_work}

This section summarizes a number of earlier initiatives to ensure the safety of
 privileged code. %in user space and kernel space.
The literature referenced in this section includes past efforts to define safety metrics,
and to quantify areas of risk in an OS kernel, as well as background information
 on technologies incorporated into Lind.

\textbf{Safety metrics.} \lois{Do we need to repeat this information that was presented in Section 3?}
As discussed in Section~\ref{sec.metric}, a number of metrics have been put forth over the years to identify where vulnerabilities might be found in code.
One such approach looked at identifying a correlation between the presence of bugs
and certain directories or modules within a system (Chou~\cite{PittSFIeld},
Palix~\cite{palix2011faults}.)
However, this approach does not necessarily
provide effective detection and defense against kernel bugs, as our test of the
Chou metric in Section~\ref{sec.metric} attests.

Other approaches to identifying kernel vulnerabilities grew out of research in
%code complexity and code comprehension in
software design.
Alhazmi~\cite{alhazmi2008application}
used metrics like defect density and fault exposure ratio to assess potential risk,
while Zimmermann~\cite{zimmermann2010searching}
 pointed towards code complexity, and code churn (the total added/modified/deleted
lines of code) to predict vulnerabilities. Kim in~\cite{kim2007vulnerability} analyzed
the correlation between shared code size and shared vulnerabilities in
successive versions of a software system. In~\cite{engler2001bugs}, Engler analyzed
system code by static analysis to look for contradictions, such as acquired locks that are
not released. While
these prior approaches can be useful for determining software readiness for release, and
evaluating the risk of if vulnerabilities are exploited,
they are less useful in providing insights for secure system design.

Lastly, prior work has also examined the evolution of kernel defects over time, including
Ozment~\cite{ozment2006milk} (see Section~\ref{sec.metric}) and Li~\cite{li2006have}.
Li found a decrease in memory-related
bugs, and identified semantic bugs as being the dominant root cause of these
defects.
% \lois{this sentence about Li says nothing and reads like
% nonsense. He reported an increase in the bugs by studying the number of bugs?
%I checked the paper but had no clue what this referred to}
%\yiwen{I fixed it. Please check.}
However, unlike Lind, neither of these approaches provides insights
on how vulnerabilities are distributed.

%Much work has been done in analyzing system faults using techniques in
%software engineering. For example,

\textbf{Virtualization systems.}
Lind incorporates a number of existing virtualization techniques, which are
described below.
%This section details some
%of the significant work relevant to the development of today's virtual machines,
%including Lind.

\textit{System Call Interposition} (SCI) tracks all the system calls of processes such
that each call can be modified or denied.
Goldberg, et al. developed Janus \cite{Janus0:96, Janus:99},
%a filtering-based SCI system. The key idea is to
which adopted a user-level ``monitor'' to filter system call requests based on
user-specified policies. Garfinkel, et al. proposed a delegating architecture for secure system call interposition
called Ostia \cite{SCI-04}. Their system introduced emulation libraries in the user space
to mediate sensitive system calls issued by the sandboxed process. SCI is similar to
 the Lind isolation mechanism. However, SCI-based tools can easily be circumvented
 if the implementation is not careful~\cite{Problems-SCI}.
% However, a key difference is the actual execution
%of a system call through Lind's SafePOSIX re-creation.
%In SCI systems, there is no such re-creation module.

\textit{Software Fault Isolation} (SFI) transforms a given program so that it can be guaranteed to satisfy a security policy.
%is another widely-used technique to build virtual machines.
Wahbe, et al. \cite{SFI:93} presented a software approach to implementing
fault isolation within a single address space.
Yee, et al. from Google developed Native Client (NaCl) \cite{NaCl-09},
an SFI system for the Chrome browser that allows native executable code to run directly in a
browser. As discussed in Section 5, Lind adopts NaCl as a key component to ensure secure execution
of binary code.

\textit{Language-based virtualization.}
Programming languages like Java, JavaScript, Lua~\cite{Lua}, and
Silverlight~\cite{Silverlight} can provide safety in virtual systems by
``translating" application commands into a native language.
%
Though many sandboxes implement the bulk of standard libraries in
memory-safe languages like Java or C\#, flaws in this code can
still pose a threat~\cite{JavaBugs, Java-Lessons}.
Any bug or failure in a programming language virtual
machine is usually fatal. In contrast, the main component of Lind
is built using Repy, which is a programming language with a very small TCB, minimizing the chance of contact with kernel flaws.

\textit{OS virtualization}
techniques include
bare-metal hardware virtualization, such as VMware ESX Server, Xen~\cite{Xen-03},
LXC~\cite{LXC}, BSD's jail, Solaris zones, and Hyper-V, and
hosted hypervisor virtualization, such as VMware
Workstation, VMware Server, VirtualPC and VirtualBox.
Security by isolation \cite{Qubes, Overshadow, SecureVM, HypSec}
provides safe executing environments through containment for multiple
user-level virtual environments sharing the same hardware.
However, this approach is limited due to
the large attack vectors against the hypervisors.

\textit{Library OSes}
allow applications to efficiently obtain the benefits of virtual machines
by refactoring a traditional OS kernel into an application library.
Porter, et al. developed Drawbridge \cite{Drawbridge-11},
a library OS
%that uses picoprocesses (lightweight containers),
%a security monitor (to enforce rules),
%and a library OS
that presents a Windows persona for %a wide variety of
Windows applications. Similar to Lind,
it restricts access from usermode to host OS through
operations that pass through the security monitor.
%\lois{How does this fit with Lind?}
%
%This approach brings many of the benefits of VM based temporal,
%spatial and fault isolation properties to a per-process level.
%
Baumann, et al. presented Bascule \cite{Bascule}, an architecture for library OS extensions
based on Drawbridge that allows application behavior to be customized by
extensions loaded at runtime. The same team also developed Haven \cite{Haven},
which uses a library OS to implement
shielded execution of unmodified server applications
in an untrusted cloud host.
Tsai, et al. developed Graphene \cite{Graphene-14}, a library OS that
executes both single and
multi-process applications with low performance overheads.

While library OSes rely extensively on
the underlying kernels to perform system functions,
Lind re-creates complex OS functions through memory-safe Repy
code, relying on only a limited set of system functions from the Repy
sandbox. Furthermore, Drawbridge, Bascule or Haven
do not have a sandbox environment to properly contain
buggy or malicious behavior, while Lind does have this more secure environment. 
