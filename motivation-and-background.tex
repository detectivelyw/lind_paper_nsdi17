\section{Goals and Threat Model}
\label{sec.motivation-and-background}

\textbf{Goals.}
Our goal is to design and build a secure virtualization system that allows
untrusted programs to run on an unpatched and vulnerable host OS (Linux OS in
this study), without triggering vulnerabilities that attackers could exploit.
Developing effective defenses for the host OS kernel is essential as kernel code
can expose privileged access to attackers that could lead to a system take-over.

To combat the threat from zero-day vulnerabilities, untrusted programs are often executed in a 
secure virtualization system, such as a guest OSVM, a system call 
interposition module, or a library OS system. Our intent is to
build such a system capable of protecting a vulnerable underlying host OS, 
while running untrusted user programs.

In this section, we define the scope of our efforts. We also briefly note why 
this study does not evaluate a few existing design schemes.

\noindent
\textbf{Threat model.}
%We start by acknowledging that 
When an attack attempt is staged
on a host OS in a virtualization system,
%Furthermore, %we anticipate that 
the exploit could be done directly or indirectly.
In a direct exploit, the attacker can access a portion of host OS's kernel
that has a vulnerability through crafted attacking code. In an indirect exploit,
the attacker first takes advantage of a vulnerability in the virtualization system itself 
(for example, a buffer-over-flow vulnerability) 
to escape the VM's containment. Then the attacker would be able to run arbitrary code 
in the host OS. 
The secure virtualization system design we propose
in Section~\ref{sec.design} can prevent both types of attacks effectively.

Based on the goals mentioned above, we make the following assumptions about the
potential threats our system could face:

\begin{itemize}\setlength\itemsep{0em}

\item The attacker possesses the knowledge about one or more unpatched 
vulnerabilities in the host OS.

\item The attacker is permitted to execute any code in the secure 
virtualization system.

\item If the attack program can trigger a vulnerability in any privileged 
code, whether in the host OS or the secure virtualization system, the attacker 
is then considered successful in compromising the system.

\end{itemize}

\noindent
\textbf{Exclusion.}
It should be noted that our study intentionally excludes %several existing
%approaches to virtualization systems. We considered these techniques
%out of scope due to differences in hardware and software requirements.
%Primarily we chose to exclude 
a comparison with solutions that do not run on top of a 
full-fledged privileged OS, such as 
%a virtualization system that uses 
a bare-metal hypervisor~\cite{Xen-03, VMWare-Server} or 
hardware-based virtualization~\cite{IntelVT, keller2010nohype}. 
While our techniques can potentially apply to those 
systems, a direct comparison is not possible since they have different 
ways of accessing hardware resources and require different measuring approaches. 
%would be difficult to perform due to 
%tracing techniques.
%These devices are dependent on the underlying hardware, and do not interact
%with a privileged OS kernel. Such differences in both structure and function
%make it difficult to directly compare these techniques to our proposed model.

In addition, we exclude evaluation and direct comparison with full virtualization virtual machines, 
such as VirtualBox \cite{VirtualBox}, VMWare Workstation \cite{VMWare-Workstation}, and QEMU \cite{QEMU}. 
For those full virtualization systems, 
the virtual machine simulates hardware to allow an unmodified guest OS to run. While the goal 
of our design is not to run a full guest OS with large and complex TCB, but to support single-process 
programs with small TCB and secure isolated environment. With different goals, our proposed design is fundamentally 
a different approach from full virtualization. The two approaches have very different TCBs, and have entirely different 
mechanisms to serve system calls to user programs. Direct measurement and comparison between full virtualization 
and our design is not fair and can be very hard to conduct. Therefore, evaluation of full virtualization virtual machines is 
beyond the scope of this work.  

To build a VM to resist zero-day vulnerabilities, we need to know which
portions of the kernel may be more prone to exploitation. Our first step is to
define and test a security metric that can quantitatively measure how bugs and
vulnerabilities are distributed in the host OS kernel.
