\section{Goals and Threat Model}
\label{sec.motivation-and-background}

In this section, we define the scope of our efforts. We also briefly note why
this study does not evaluate a few existing design schemes.

\textbf{Goals.}
Our goal is to design and build a secure virtualization system that allows
untrusted programs to run on an unpatched and vulnerable host OS (Linux OS in
this study), without triggering vulnerabilities that attackers could exploit.
Developing effective defenses for the host OS kernel is essential as kernel code
can expose privileged access to attackers that could lead to a system take-over.

To combat the threat from zero-day vulnerabilities, untrusted programs are often executed in a
secure virtualization system, such as a guest OSVM, a system call
interposition module, or a library OS system. Our intent is to
build such a system capable of protecting a vulnerable underlying host OS,
while running untrusted user programs.


\noindent
\textbf{Threat model.}
%We start by acknowledging that
When an attack attempt is staged
on a host OS in a virtualization system,
%Furthermore, %we anticipate that
the exploit can be done either directly or indirectly.
In a direct exploit, the attacker accesses a vulnerable portion of the host OS's kernel
using a crafted attack code. In an indirect exploit,
the attacker first takes advantage of a vulnerability in the virtualization system itself
(for example, a buffer-over-flow vulnerability)
to escape the VM's containment. Once past the containment, the attacker would be able to run arbitrary code
in the host OS.
The secure virtualization system design we propose
in Section~\ref{sec.design} can prevent both types of attacks effectively.

Based on the goals mentioned above, we make the following assumptions about the
potential threats our system could face:

\begin{itemize}\setlength\itemsep{0em}

\item The attacker possesses knowledge of one or more unpatched
vulnerability in the host OS.

\item The attacker can execute any code in the secure
virtualization system.

\item If the attack program can trigger a vulnerability in any privileged
code, whether in the host OS or the secure virtualization system, the attacker
is then considered successful in compromising the system.

\end{itemize}

\noindent
\textbf{Exclusion.}
It should be noted that our study intentionally excludes %several existing
%approaches to virtualization systems. We considered these techniques
%out of scope due to differences in hardware and software requirements.
%Primarily we chose to exclude
a comparison with solutions that do not run on top of a
full-fledged privileged OS, such as
%a virtualization system that uses
a bare-metal hypervisor~\cite{Xen-03, VMWare-Server} or
hardware-based virtualization~\cite{IntelVT, keller2010nohype}.
While our techniques can potentially apply to those
systems, a direct comparison is not possible since they have different
ways of accessing hardware resources, and require different measuring approaches.
%would be difficult to perform due to
%tracing techniques.
%These devices are dependent on the underlying hardware, and do not interact
%with a privileged OS kernel. Such differences in both structure and function
%make it difficult to directly compare these techniques to our proposed model.

In addition, we exclude evaluation and direct comparison with full virtualization virtual machines,
such as VirtualBox \cite{VirtualBox}, VMWare Workstation \cite{VMWare-Workstation}, and QEMU \cite{QEMU}.
Such systems simulate hardware to allow an unmodified guest OS to run. The goal
of our design is to substitute the large and complex TCB required for a guest OS, with a single-process
program with a small TCB and a secure isolated environment. With different goals, our proposed design is 
a fundamentally different approach from full virtualization. As a result, direct measurement and comparison between full virtualization
and our design is
beyond the scope of this work.

To build a VM to resist zero-day vulnerabilities, we need to know which
portions of the kernel may be more prone to exploitation. Our first step is to
define and test a security metric that can quantitatively measure how bugs and
vulnerabilities are distributed in the host OS kernel.
