\section{Goals and Threat Model}
\label{sec.motivation-and-background}

\textbf{Goals.}
Our goal is to design and build a secure virtualization system that allows
untrusted programs to run on an unpatched and vulnerable host OS (Linux OS in
this study), without triggering vulnerabilities that attackers could exploit.
Developing effective defenses for the host OS kernel is essential as kernel code
can expose privileged access to attackers that could lead to a system take-over.

To combat the threat from zero-day vulnerabilities, untrusted programs are often executed in a 
secure virtualization system, such as a guest OS virtual machine, a system call 
interposition module, or a library OS system. Our intent is to
build such a system capable of protecting not only
an underlying host OS, but also any programs running in it. 
%from both malicious attacks
%and accidental triggering by applications.  \cappos{I do not know what
%you mean to convey with the previous sentence.}
In this section, we define the scope of our efforts. We also briefly note why 
this study does not evaluate a few existing design schemes.

\noindent
\textbf{Threat model.}
\cappos{This paragraph also is confusing to me.}
\yanyan{please check my edits.}
%We start by acknowledging that 
When an attack attempt is staged
on a host OS in a secure virtualization system,
%Furthermore, %we anticipate that 
the exploit could be done directly or indirectly.
In a direct exploit, the attacker can access a portion of host OS's kernel
that has a vulnerability. In an indirect exploit,
the attacker takes advantage of a vulnerability in the virtualization system to 
run arbitrary code, escapes the VM's containment and  
makes system calls into the host OS. 
The secure virtualization system design we propose
in Section~\ref{sec.design} can prevent both types of attacks.


Based on the goals mentioned above, we make the following assumptions about the
potential threats our system could face:

\begin{itemize}\setlength\itemsep{0em}

\item The attacker possesses the knowledge about one or more unpatched 
vulnerabilities in the host OS.

\item The attacker is permitted to execute any code in the secure 
virtualization system.

\item If the attack program can trigger a vulnerability in any privileged 
code, whether in the host OS or the secure virtualization system, the attacker 
is then considered successful in compromising the system.

\end{itemize}

\noindent
\textbf{Exclusion.}
It should be noted that our study intentionally excludes %several existing
%approaches to virtualization systems. We considered these techniques
%out of scope due to differences in hardware and software requirements.
%Primarily we chose to exclude 
a comparison with solutions that do not run on top of a 
full-fledged privileged operating system, such as 
%a virtualization system that uses 
a bare-metal hypervisor~\cite{Xen-03, VMWare-Server} or 
hardware-based virtualization~\cite{IntelVT, keller2010nohype}. 
While our techniques can potentially apply to those 
systems, a direct comparison is not possible since they have different
trusted computing bases.
%would be difficult to perform due to 
%tracing techniques.
%These devices are dependent on the underlying hardware, and do not interact
%with a privileged OS kernel. Such differences in both structure and function
%make it difficult to directly compare these techniques to our proposed model.

To build a system to resist zero-day vulnerabilities, we need to know which
portions of the kernel may be more prone to exploitation. Our first step is to
define and test a security metric that can quantitatively measure how bugs and
vulnerabilities are distributed in the kernel.
