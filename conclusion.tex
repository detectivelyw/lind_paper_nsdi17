\section{Conclusion}
\label{sec.conclusion}

In this paper, we propose a new security metric based on quantitative measures of kernel code execution when running user applications.
After determining that fewer bugs exist in popular paths associated with frequently-used
programs, we devise a new design for secure virtual machines, called \lip.
As the name implies, the design scheme locks access to all
kernel code except that found in paths frequently used by
popular programs. We test the \lip idea by implementing a prototype system
called Lind, which features a minimized TCB and prevents direct access to application
calls from less-used, riskier paths.
Instead, Lind supports complex system calls by securely re-creating
essential OS functionalities inside a sandbox.
In tests against Docker, LXC, and Graphene, Lind emerged as the most effective system in preventing
zero-day Linux kernel bugs.

So that other researchers may replicate our results, we make all of the kernel
trace data, benchmark data, and source code for this paper available. 
\cappos{We need to anonymize this link and others that refer to us in the wrong way...} 
\yiwen{Is it Okay to remove the link?}
