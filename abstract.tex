\subsection*{Abstract}

Virtual machines (VMs) have been widely used in practice, in part for their ability to
isolate potentially untrusted code from the rest of a system. 
Recently, library OSes and containers became promising approaches.  
%
However, it is often possible to trigger zero-day flaws
in the host Operating System(OS) from inside of such virtualized systems. 
%
In this paper, we use a new insight about where security bugs lie and later use
this to devise a design to improve the security of applications
run in VMs on top of a vulnerable host OS. 
We begin by observing that a portion of the OS kernel, the kernel paths accessed
by popular applications in everyday use, contains fewer security bugs than less-used paths. We
leverage this observation to devise the \lip design, which
locks an application, and the POSIX implementation that services it, into
accessing only the well-used popular portion of the kernel.  Using the \lip model, we
implement a virtual machine called Lind. 
\yiwen{Maybe we should call Lind a library OS.}
%
We compare Lind and three other virtualized systems that were
available at the release of Linux kernel version 3.14.1, and evaluate
their effectiveness in containing the zero-day kernel bugs that have been discovered
since then.
%
Our results show that Lind can prevent the triggering of zero-day kernel bugs significantly better 
than existing library OS Graphene, and containers such as Docker and LXC. 
