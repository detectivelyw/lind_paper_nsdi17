\subsection*{Abstract}

Virtual Machines(VMs) are widely used in practice, in part for their ability to
isolate potentially untrusted code from the rest of a system.
However, it is often possible to trigger zero-day flaws
in the host Operating System(OS) from inside of a guest OS. 
%
In this paper, we use a new insight about where security bugs lie to devise a design to improve the security of applications
run in VMs on top of a vulnerable host OS. 
We begin by observing that a portion of the OS kernel, the kernel paths accessed
by popular applications in everyday use, contains fewer security bugs than less-used paths. We
leverage this observation to devise the \emph{Lock-in-Pop} design, which
locks an application, and the POSIX implementation that services it, into
accessing only the well-used popular portion of the kernel.  Using the \emph{Lock-in-Pop} model, we
implement a virtual machine called Lind.
%
We compare Lind and seven other virtualization systems that were
available at the release of Linux kernel version 3.14.1, and evaluate
their effectiveness in containing the zero-day kernel bugs that have been discovered
since then.
%
Our results show that it is substantially more difficult to trigger zero-day bugs in
Lind than existing systems like VirtualBox, VMWare Workstation, 
Docker, LXC, QEMU, KVM, and Graphene.