\section{Discussion}
\label{sec.limitation}

One of our challenges in conducting this study was deciding where to place the
limits of its scope.  To explore any one strategy
in depth, we felt it was necessary to intentionally exclude consideration of
a few other valid approaches. These choices may have placed some limitations on our results.

\textbf{Limitation }
A primary limitation of our work stems from our chosen criteria for locating
bugs. At the beginning
of our study, we identified a set of common, but seriously dangerous, zero-day bugs
and then we looked for them in our obtained kernel traces. By looking only
for a specific subset of bugs, we might have limited our
ability to find a broader spectrum of kernel vulnerabilities. For example, bugs
caused by a race condition, or that involve defects in the internal kernel data
structures, or that require complex triggering conditions across multiple kernel
paths, may not be immediately found using our metric. As we continue to refine
our metric, we will look to also evolve our evaluation
criteria to find and protect against more complex types of bugs.
In the meantime, avoiding the triggering of this initial set of bugs
through the use of our \lip design can address the security
needs of a significant segment of users.

\textbf{Future work.}
While our experiments were limited to the Linux kernel 3.14.1, our future work
will include testing its applicability to other operating systems, such as
Windows and Mac OS. Since \lip is not dependent on the use of any
specific hardware, we believe it can be adapted to these other
widely-used systems.

The other challenge facing wide-scale adoption of Lind is improving its
performance in terms of bandwidth and other overhead factors. As addressed in
Section 6.4, Lind does incur some performance overhead. Future work will focus on identifying
the factors that contribute to this overhead, and the best ways to make \lip
a cost-effective alternative to other virtual machine designs.
