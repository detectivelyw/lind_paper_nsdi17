\section{Introduction}
\label{sec.introduction}

%\cappos{Paragraph: Bad guys (hackers, NSA, etc.) are using zero day bugs in
%the kernel to do bad things...}
%For many years system designers and developers have had to deal with a common security issue --
%how to defend against threats hidden within the system's own code. Despite
% decades of effort, researchers still frequently uncover new flaws in operating
% system kernels. Such flaws are dangerous because, if triggered by an untrusted
% program, these bugs can lead to such security threats as privilege escalation
% [CVE-2016-0728], [CVE-2015-8660], denial of service [CVE-2015-8539], [CVE-2015-5364],
%  and memory corruption [CVE-2014-9529].
%
The number of attacks in which zero-day vulnerabilities have been exploited has more than
doubled from 2014 to 2015~\cite{zero-day}. Skilled hackers can find a security
flaw in a system and use it to hold the system's users hostage , e.g., by
gaining root access and compromising the host~\cite{linux-0day}. Similarly, zero-day
vulnerabilities can be exploited~\cite{fbi-0day} or held back~\cite{nsa-0day} by government
agencies, thus rendering millions of devices vulnerable.
%for national security reasons
%\cappos{Must be OS kernel zero days.  Firefox
%doesn't matter and privacy isn't the biggest concern.  It's someone breaking
%into systems due to a flaw like this.}
%It is difficult to defend against these new and
%unknown vulnerabilities, leading to an arms race between hackers and software
%developers.
%

In theory, running a program in an operating-system-level virtual machine (OSVM) like Docker or LXC should
prevent it from triggering bugs in the host OS kernel.  However, to be
effective, the isolation provided by such systems must meet two challenging criteria. First, the OSVM's
software must not contain any bugs that could allow the program to escape the
machine's containment and interact directly with the host OS.
Unfortunately, these issues are very common in OSVMs. Virtualbox reports more than 40 such
vulnerabilities~\cite{Virtualbox-Vulnerabilities} and more than 100 bugs have
been found in VMware Workstation~\cite{VMWare-Vulnerabilities}.  Given the large amount of complex code needed for such a system, it is understandable that
flaws could occur,
leaving tens of millions of user machines at risk~\cite{linux-0day}.

Secondly, isolation will not work if a malicious
program can access a portion of the host OS's kernel that contains a
zero-day flaw. This may occur even if the containment of the OSVM is
working as designed.
Many system calls made within an OSVM eventually
result in system calls in the host OS (e.g., network I/O from the guest
OSVM results in network I/O by the host OS kernel). If one of these paths in
the OS kernel contains a zero-day security bug, the attacker
may be able to trigger and exploit it.

%aspect of this isolation is preventing the program from triggering bugs
%in the host OS kernel or in the OSVM itself.  However,

%The search for an ideal way to protect kernel code has been hampered by limited
%knowledge of the security impact of its interaction with user programs.
%A number of strategies have been attempted. One prevailing belief was that virtual
%machines could provide enough isolation to execute programs safely.
%However, when Virtualbox reports 40 vulnerabilities \cite{Virtualbox-Vulnerabilities},
% and more than 100 bugs have been found in VMware Workstation,
%it suggests that isolation alone may not be enough. Other researchers have looked
% to the development of metrics capable of pinpointing bugs, and,
%securing just those vulnerable areas\cite{PittSFIeld, ozment2006milk}. As these
%examples indicate, the design of secure systems has been somewhat, hit-or-miss.�
% Observations may lead to
% the reduction of specific threats, but not to a reproducible, broadly applicable,
%quantitative design solution.

With this paper, we move one step closer to designing secure OSVM systems that are
resilient to zero-day flaws.
We start with the proposition that kernel code found in popular paths, associated with frequently-used programs,
has less potential risk of bugs than code in less-used parts of the kernel.
Our intuition behind this proposition is that bugs in the popular paths are
more frequently found in software testing, because of the numerous times they are executed by
diverse pieces of software.
We performed a quantitative analysis of resilience
to flaws in two versions of the Linux kernel (version 3.13.0 and version 3.14.1), and
found that only about 3\% of the bugs were present in popular code paths,
despite these paths accounting for about one third of the total reachable
kernel code.
Therefore, OSVMs that only use these kernel paths will
greatly increase resilience to zero-day bugs in the host OS kernel.
%When we ran the same study on the same Linux kernel versions
%using two other metrics
%(Chou \cite{PittSFIeld} and Ozment \cite{ozment2006milk}),
%we found them less effective at predicting the location of zero-day bugs.

Guided by this knowledge, we propose a new design scheme for a secure virtual machine that
accesses only popular code lines through a very small trusted computing base.
We name it \lip, as it locks kernel access to only code
associated with popular programs. %However, the popular kernel functionality is
%not sufficient to implement all of the system call functionality needed by
%applications.
However, there are many system calls outside of the popular paths that need to
be supported to run applications~\cite{tsai2016study}.  Our second
observation in \lip is to re-create needed-but-risky OSVM functionality
(e.g., POSIX) in a sandbox that only allows access to popular kernel paths.
In this way a bug inside of the OSVM implementation does not result in a system compromise
so long as the minimal sandbox remains uncompromised.

%This is sufficient to
%perform operations for file systems, such as
%networking, threading, memory management, and other required functions.
%\cappos{The fact it is a PL sandbox is an implemenation detail, not design
%detail...  Comment out the next sentence if you agree.}
%For complex functionalities,  \emph{Lock-in-Pop} re-creates system calls in
%a memory-safe programming language sandbox.

To demonstrate the viability of \lip, we use it to implement a
prototype virtual machine that can offer enhanced security without sacrificing
basic functionality. Dubbed Lind, it pairs two components -- Google's Native Client
(NaCl) and Seattle's Repy. \lois{Do we want to mention those two products. I thought the functions were important but not the particular brand of component chosen, since the system is supposed to be flexible} NaCl serves as a computational module that isolates
binaries, providing memory safety for legacy programs running in our OSVM.
It also passes system calls invoked by the program to the operating system interface, called SafePOSIX.
SafePOSIX re-creates the broader POSIX functionalities needed by applications.
SafePOSIX does this while being contained within the Repy sandbox, which
provides an API that only allows access to popular kernel paths.
The small (8K LOC) sandbox kernel of Repy isolates flaws in SafePOSIX
to prevent them from allowing direct access to the host OS kernel.

To test Lind's effectiveness, we replicated 35 kernel bugs that were
discovered in the Linux kernel version 3.14.1.  We attempted to trigger those
bugs in three other virtualized environments,
including Docker, LXC, and Graphene.
Our results show that applications in Lind were substantially less likely to trigger
kernel bugs.

In summary, the main contributions of this paper are as follows:

\begin{itemize}\setlength\itemsep{0em}
%\item
%We postulate a new approach for securing privileged code,
%such as the OS kernel, based on the idea that popular kernel paths contain
%fewer bugs.  \cappos{I'm not sure how this differs from 3...}

\item
We propose a quantitative metric that evaluates security at the line-of-code
level, and test its effectiveness against other proposed metrics,
such as the age of code~\cite{ozment2006milk}, and
the increased risk in driver code~\cite{PittSFIeld}.  We find that =
choosing popular kernel paths is a much more effective metric for identifying lines of
code that are unlikely to contain flaws.
%of the kernel that
%After testing this metric against others, we find it effective in locating
% bugs in the Linux kernel.

\item
Based on the metric, we postulate a new approach for securing privileged code,
and develop a new design scheme called \lip. It
accesses only popular code paths
through a very small trusted computer base.
The need for complex functionality is addressed by re-creating riskier systems calls
in a memory-safe programming language within a secure sandbox.

\item
We build a prototype virtual machine, Lind, \lois{we need to be consistent in how we describe Lind. Is it a virtual machine or a Library OS machine, as its called in the abstract} using the \lip design,
and test its effectiveness against three other security systems. We find that
Lind exposes 5-14x fewer kernel code paths containing bugs than prior production and
academic systems.
\end{itemize}

The remainder of this paper is organized as follows.
Section \ref{sec.motivation-and-background} presents the scope of our study
and precisely describes our threat model.
%To understand which kernel paths a
Earlier kernel protection metrics and how they performed
against our newly proposed metric are discussed in Section \ref{sec.metric}.
We will discuss handling bugs in the OSVM software in
Section~\ref{sec.design}, while focusing on design strategies for preventing exploitation of zero-day kernel
bugs in the host OS.
Section \ref{sec.design} also describes our \lip design scheme.
In Section \ref{sec.implementation} we discuss the construction of the Lind
prototype, while Section \ref{sec.evaluation} provides a quantitative
security analysis of Lind compared to other virtualization systems.
Section \ref{sec.limitation} outlines study limitations and possible future initiatives.
Finally, Section \ref{sec.related_work} reviews existing work relevant to Lind's security goals,
while the cogent points relayed in the paper are reviewed in Section \ref{sec.conclusion}.
